\emph{Фазоманипулированный сигнал} "--- это последовательность элементарных импульсов длительностью $\tau_0$, фазы сигнала в которых изменяются по определенному закону, принимая два или более фиксированных значений. При этом, если количество элементарных импульсов равно $N$, то общая длительность сигнала составляет $T=\tau_0\cdot N$. Ширина спектра такого сигнала примерно равна \cite{Varakin}.

\begin{equation}
\Delta f= \frac{1}{\tau_0} = \frac{N}{T}.
\label{lab5_eq:spectr}
\end{equation}
	
Фазоманипулированный сигнал является сложным, а его база с учетом выражения \eqref{lab5_eq:spectr} определяется как $B = \Delta f \cdot T = N$.
	
Опишем сигнал несущей частоты выражением

\begin{equation*}
S(t) = A\cos(2\pi f t +\theta), 
\label{lab5_eq:carryf}
\end{equation*}
где $A$ "--- амплитуда, $f$ "--- частота, $\theta$ "--- начальная фаза несущего колебания.
   
Тогда выражение для фазоманипулированного сигнала будет иметь вид

\begin{equation}
S_m(t) = A\cos(2\pi f t +\theta_k(t)),\; k=\overline{1, N}, 
\label{lab5_eq:fkm}  
\end{equation}	
где $k$"--- порядковый номер импульса.

В пределах каждого импульса начальная фаза сигнала не изменяется, а в начале каждого нового импульса принимает одно из дискретных значений, определяемых выражением 

\begin{equation}
\theta_j= \frac{\pi(2j-1)}{J},\;\; j=\overline{1, J}, 
\label{lab5_eq:phase}
\end{equation}
где $j$"--- порядковый номер значения из дискретного множества начальных фаз сигнала, $J$"--- общее количество элементов множества начальных фаз.	

В зависимости от значения $J$ из формулы \eqref{lab5_eq:phase} различают виды фазовой манипуляции: двоичная ($J = 2$), квадратурная ($J = 4$), восьмеричная ($J = 8$). В современных системах связи данные виды манипуляции обозначают также BPSK, QPSK и 8-PSK соответственно. При этом количество бит $B$, кодируемых одним импульсом определяется выражением

\begin{equation}
B = log_2 J. \label{lab5_eq:numbits}
\end{equation}
	
Из выражения~\eqref{lab5_eq:numbits} следует, что с увеличением $J$ ёмкость закодированной информации, а для систем цифровой связи и теоретическая скорость передачи данных, возрастает. Теоретически $J$ может принимать значения и больше $8$, тогда в общем виде модуляцию можно обозначить как M-PSK. На практике в системах цифровой связи с уменьшением соотношения сигнал-шум или с увеличением $J$ на стороне приемника становится сложнее различать значения фаз. 
	
Для обеспечения помехозащищённости при обработке сигнала средствами согласованной фильтрации значения $\theta_k$ фазоманипулируемого сигнала задают в соответствии с псевдослучайными последовательностями, так как их автокорреляционные функции обеспечивают лучшее соотношение уровней основного и боковых лепестков при малой ширине основного лепестка. На практике часто применяют последовательность Баркера и M"~последовательности.
		
\emph{Последовательность Баркера} "--- это числовая последовательность $a_1, a_2, \ldots a_N,$ где каждый элемент равен $+1$ или $-1$~\cite{Dachnovich}, причем
$$
	\left| \sum\limits_{j=1}^{N-v} a_j a_{j+v} \right| \leq 1,\,\forall v:1 \leq v < N.
$$
	
В настоящее время известны девять последовательностей Баркера (\autoref{lab5_tab:table_1}), а наиболее длинная из них имеет длину $N_{max}=13$. Соотношение уровней основного и боковых лепестков автокорреляционной функции зависит от длины последовательности Баркера и равно $\displaystyle \frac{1}{N}$.

\begin{table}[H]
\caption{Последовательности Баркера}\label{lab5_tab:table_1}
\begin{tabular}{|p{1.2cm}|p{4cm}|p{4cm}|}
\hline
Длина & \multicolumn{2}{c|}{Последовательности} \\ \hline
$2$ & $+1 -1$ & $+1 +1$ \\ \hline
$3$ & \multicolumn{2}{l|}{$+1 +1 -1$} \\ \hline
$4$ & $+1 -1 +1 +1$ & $+1 -1 -1 -1$ \\ \hline
$5$ & \multicolumn{2}{l|}{$+1 +1 +1 -1 +1$} \\ \hline
$7$ & \multicolumn{2}{l|}{$+1 +1 +1 -1 -1 +1 -1$} \\ \hline
$11$ & \multicolumn{2}{l|}{$+1 +1 +1 -1 -1 -1 +1 -1 -1 +1 -1$} \\ \hline
$13$ & \multicolumn{2}{l|}{$+1 +1 +1 +1 +1 -1 -1 +1 +1 -1 +1 -1 +1$} \\ \hline
\end{tabular}
\end{table} 
	
\emph{М"~последовательность} "--- это псевдослучайная двоичная последовательность максимальной длины, порожденная регистром сдвига с линейной обратной связью и имеющая максимальный период~\cite{Varakin}. 
	
Для формирования M"~последовательностей различной длины используют порождающие полиномы. В \autoref{lab5_tab:table_2} приведены полиномы $h(x)$ до $n=11$ включительно, которые могут быть использованы для генерации соответствующих М"~последовательностей~\cite{Mak}.

   	\begin{table}[H]
   	\caption{Полиномы, используемые для генерации соответствующих М"~последовательностей} 
	\label{lab5_tab:table_2}
		\begin{tabular}{|c|p{5.1cm}|c|}
			\hline
			\multicolumn{1}{|c|}{n} & \multicolumn{1}{c|}{$h(x)$} & \multicolumn{1}{c|}{$M=2^n-1$} \\ \hline
			$1$ & $x+1$ & $1$ \\ \hline
			$2$ & $x^2+x+1$ & $3$ \\ \hline
			$3$ & $x^3+x+1$ & $7$ \\ \hline
			$4$ & $x^4+x+1$ & $15$ \\ \hline
			$5$ & $x^5+x^2+1$ & $31$ \\ \hline
			$6$ & $x^6+x+1$ & $63$ \\ \hline
			$7$ & $x^7+x+1$ & $127$ \\ \hline
			$8$ & $x^8+x^6+x^5+x+1$ & $255$ \\ \hline
			$9$ & $x^9+x^4+1$ & $511$ \\ \hline
			$10$ & $x^{10}+x^3+1$ &  $1023$ \\ \hline
			$11$ & $x^{11}+x^2+1$ & $2047$ \\ \hline
		\end{tabular}
\end{table}
	
М"~последовательности обладают следующими основными свойствами~\cite{Golomb}:	
\renewcommand{\labelitemi}{--}
\begin{itemize}
    \item М"~последовательности являются периодическими с периодом $M=2^n-1$, где $n$"--- порядок порождающего полинома $h(x)$;
    \item количество символов, принимающих значение единица, на длине одного периода М"~последовательности на единицу больше, чем количество символов, принимающих значение ноль;
    \item сумма по модулю $2$ любой М"~последовательности с её произвольным циклическим сдвигом также является М"~последовательностью;
    \item периодическая автокорреляционная функция любой М"~последовательности имеет соотношение уровней основного и боковых лепестков, равное $\displaystyle\frac{1}{M}$;
    \item автокорреляционная функция усечённой М"~последовательности (непериодическая последовательность длиной, соответствующей периоду $M$) имеет соотношение уровней основного и боковых лепестков, близкое к $\displaystyle\frac{1}{\sqrt{M}}$.
\end{itemize}
\vspace{10mm}