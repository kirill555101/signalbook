\documentclass[10pt]{book}
\usepackage{bmstu-lab-book}
\usepackage{forloop}
\usepackage[subpreambles=false]{standalone}
\usepackage[pages=some]{background}
\usetikzlibrary{decorations.pathreplacing}
\usetikzlibrary{arrows,shapes,positioning,shadows,trees}
\usepackage{titlepic}
\usepackage{graphicx}
\usepackage{lastpage}
\usepackage{fancyhdr}
\usepackage{rotating}
\usepackage[font=small]{caption}


%%% Интервал между абзацами, дополняющий обычный интервал между строками
\setlength{\parskip}{0mm}
% Настройка для отображения ответов на контрольные вопросы
\boolfalse{answers} 

% Знаки TradeMark и Registered
\usepackage{textcomp}
\usepackage{xspace}
\let\OldTextregistered\textregistered
\renewcommand{\textregistered}{\OldTextregistered\xspace}%
\let\OldTexttrademark\texttrademark
\renewcommand{\texttrademark}{\OldTexttrademark\xspace}%

% Для регистрации папок с конкретными лабами и определения их порядка в сборнике
\usepackage{etoolbox} 
\newcounter{cnt} 
\newcommand\textlist{} 
\newcommand\settext[2]{% 
\csdef{text#1}{#2}} 
\newcommand\addtext[1]{% 
\stepcounter{cnt}% 
\csdef{text\thecnt}{#1}} 
\newcommand\gettext[1]{% 
\csuse{text#1}} 
\newcounter{colnum} \newcommand\maketabularrow[1]{% 
\setcounter{colnum}{0}% 
\whileboolexpr { test {\ifnumcomp{\value{colnum}}{<}{#1}} }% 
{&\stepcounter{colnum}\thecolnum} } 

% Параметры страниц
\special{papersize=170mm,240mm}
\textheight 170mm % 200-(12+25)*0.35146 = 186.99598
\textwidth 120mm
\headheight13.6pt % = 0.48 mm
\oddsidemargin -10mm
\evensidemargin -10mm
\topmargin -20mm
\usepackage{setspace}
\singlespacing

%%% Для формирования списка литературы BibLaTeX (соритировка по мере появления в тексте)
\usepackage[backend=biber,bibencoding=utf8,sorting=none,maxcitenames=2,bibstyle=numeric,citestyle=numeric]{biblatex}
\addbibresource{lib.bib} % список литературы

\graphicspath{tikz}

% Чтобы ушел заголовок содержания с верхнего колонтитула каждой страницы
% Подробнее: https://www.overleaf.com/learn/latex/Headers_and_footers
\pagestyle{myheadings}

%\usepackage{setspace}\onehalfspacing
%\AtBeginDocument{%
%	\addtolength\abovedisplayskip{1\baselineskip}%
%	\addtolength\belowdisplayskip{-0.1\baselineskip}%
%	\addtolength\abovedisplayshortskip{-0.1\baselineskip}%
%	\addtolength\belowdisplayshortskip{-0.1\baselineskip}%
%}

\usepackage{tocloft}
\setlength{\cftpartindent}{0em}
\setlength{\cftchapindent}{0em}
\setlength{\cftsecindent}{0em}
\setlength{\cftsubsecindent}{0em}
\setlength{\cftsubsubsecindent}{0em}

\fancyhf{} % clear all header and footers
\renewcommand{\headrulewidth}{0pt} % remove the header rule
\fancyfoot[LE,RO]{\thepage} % Left side on Even pages; Right side on Odd pages
\pagestyle{fancy}

\renewcommand{\chapter}[2]{}% убираем стандартный заголовок для библиографии

%%% Объем данных на диске
\newcommand{\filesize}{18,5}

\begin{document}
\renewcommand{\labelitemi}{--}
\renewcommand\contentsname{}% убираем стандартный заголовок для оглавления

% Чnобы не было разрывов после формул
\setlength{\abovedisplayskip}{1.5pt}
\setlength{\belowdisplayskip}{1.5pt}

% Каждая папка с соответствующей лабой должна называться <<Lab-[label]>>, 
% где \addtext{[label]} --- регистрация лабы в сборнике
% Порядок лабораторных работ в сборнике:
\addtext{ma} 	    % 1 - moving average
\addtext{ssf}       % 2 - Synchronous signal filtering
\addtext{fir}    	% 3 - Finite impulse response filtering
\addtext{mf} 	    % 4 - Matched-filtering
\addtext{goertzel} 	% 5 - Goertzel algorithm
\addtext{fq} 	    % 6 - Frequency shifting
\addtext{fmmod} 	% 7 - Phase-Code-Manipulated Signal Receiver
\addtext{hht} 		% 8 - Hilbert–Huang transform

\begin{center}%
\section*{Оглавление}
\end{center}%ы
\tableofcontents
\thispagestyle{fancy}
%Введение

\clearpage
	\addcontentsline{toc}{section}{Введение в основы теории цифровой обработки сигналов}
\begin{center}
	\section*{Введение \\ в оcновы теории цифровой обработки сигналов}
\end{center}

\emph{Сигнал} "--- это физический процесс, являющийся средством переноса информации~\cite{Suzev}. 
Если о сигнале заранее неизвестно абсолютно ничего, то его нельзя  принять.
Если о сигнале заранее известно все, то его не нужно принимать.

В окружающем нас мире существуют всевозможные сигналы различной формы и природы происхождения. Часть сигналов являются естественными, а часть сигналов создана человеком. Сигналы окружают нас повсюду. Они исходят от радиопередатчиков, телевизоров, смартфонов и радаров "--- это лишь малая часть источников. Оповещения смартфона, звуковые сигналы автомобилей, сообщения на табло вокзала, данные, передающиеся по высокоскоростным сетям. Некоторые сигналы обеспечивают нашу жизнедеятельность (речь, жесты или мимика человека), некоторые приносят удовольствие (музыка, фильмы), а некоторые нежелательны в какой-то конкретной ситуации. В контексте информационно-управляющих систем сигналы являются носителями информации от датчиков к вычислительной подсистеме, от вычислительной подсистемы к исполнительным устройствам, при сетевом взаимодействии вычислительных подсистем между собой. В электрической системе примерами таких сигналов могут быть напряжение, ток, количество заряда. В механической системе "--- координаты положения объекта, его скорость или масса. В финансовой системе сигналами может являться цена акции, процентная ставка или обменный курс. 

Один и тот же сигнал в зависимости от поставленной перед разработчиком задачи может нести полезную информацию, то есть быть \emph{целевым} или наоборот затруднять приём информации, то есть представлять собой \emph{шум} (в случае естественного происхождения сигнала) или \emph{помеху} (в случае искусственного происхождения сигнала).


Пример интеграции изображения, полученного с помощью пакета Tikz (рисунок~\ref{labN_fig:probability}).

\begin{figure}[pt!]
	\centering
	\scalebox{.85}{\begin{tikzpicture}[
declare function= {
    triangle(\x,\a,\b,\c) = 0+(\x<=\c)*(\x>=\a)*(pow(\x-\a,2)/((\b-\a)*(\c-\a)))+(\x<=\b)*(\x>\c)*(1-pow(\b-\x, 2)/((\b-\a)*(\b-\c)))+1*(\x>\b);
}
]
\begin{axis}[
    clip=false,
    axis lines = middle,
    axis line style={shorten >=-10pt, shorten <=-10pt},
    xlabel = {$x$}, % подпись оси x
    ylabel = {$F_{x_0}(x)$}, % подпись оси y
    xlabel style={below right},
    ylabel style={above left},	
    ymax = 1.03,
    ymin = -0.07,
    xmin = -2.5,
    xtick={0}, 
    ytick={0}
    ]
    \coordinate (a1)  at (0,{triangle(3.5, 0, 9, 4)});
    \coordinate (a2)  at (3.5,{triangle(3.5, 0, 9, 4)});
    \coordinate (a3)  at (3.5,0);
    \node at (a1) [left] {$F(x_A)$};
    \node at (a3) [below] {$x_A$};
    \draw [thick, dashed] (a1)--(a2)--(a3);
    %
    \coordinate (b1)  at (0,{triangle(5, 0, 9, 4)});
    \coordinate (b2)  at (5,{triangle(5, 0, 9, 4)});
    \coordinate (b3)  at (5,0);
    \node at (b1) [left] {$F(x_B)$};
    \node at (b3) [below] {$x_B$};
    \draw [thick, dashed] (b1)--(b2)--(b3);
    %
    \addplot[domain=-3:10, samples=300, color=blue, thick] {triangle(x, 0, 9, 4)};
    %\addplot[blue,samples=200]{triangle(x, 0, 9, 4)};
    \addplot[blue, only marks,samples at={3.5,5}]{triangle(x, 0, 9, 4)};
    \addplot[domain=0:10, samples=300, color=black, thick, dashed] {1};
    \coordinate (c1)  at (0,1);
    \node at (c1) [left] {\scriptsize $1$};
\end{axis}
\end{tikzpicture}}
	\caption{Пример функции распределения непрерывной случайной величины}
	\label{labN_fig:probability}
\end{figure}


%%%% СТИЛЬ ОФОРМЛЕНИЯ КОДА %%%

%%%% Пример оформления определения
%
%\emph{Фазоманипулированный сигнал} "--- это последовательность элементарных импульсов длительностью $\tau_0$, фазы сигнала в которых изменяются по определенному закону, принимая два или более фиксированных значений. При этом, если количество элементарных импульсов равно $N$, то общая длительность сигнала составляет $T=\tau_0\cdot N$. 
%
%
%%%% Как ссылаемся на источник
%Ширина спектра такого сигнала примерно равна~\cite{Suzev}.
%
%%%% Как оформляем формулы 
%%%% labN_eq:spectr- Имя метки состоит из двух частей, где labN - номер лабы (берем из таблицы), spectr - краткое название формулы
%
%\begin{equation}
%\Delta f= \frac{1}{\tau_0} = \frac{N}{T}.
%\label{labN_eq:spectr}
%\end{equation}
%
%Фазоманипулированный сигнал является сложным, а его база с учетом выражения \eqref{labN_eq:spectr} определяется как $B = \Delta f \cdot T = N$.
%
%%%% Пример с формулой без ссылки по тексту (работает правило, описанное в Лекции 2 слайд 10)
%
%Опишем сигнал несущей частоты выражением
%
%\begin{equation}
%S(t) = A\cos(2\pi f t +\theta), 
%\label{labN_eq:carryf}
%\end{equation}
%где $A$ "--- амплитуда, $f$ "--- частота, $\theta$ "--- начальная фаза несущего колебания.
%
%В настоящее время известны девять последовательностей Баркера (\autoref{labN_tab:table1}), а наиболее длинная из них имеет длину $N_{max}=13$. Соотношение уровней основного и боковых лепестков автокорреляционной функции зависит от длины последовательности Баркера и равно $\displaystyle \frac{1}{N}$.
%
%\begin{table}[H]
%	\caption{Последовательности Баркера}
%	\label{labN_tab:table1}
%	\begin{tabular}{|p{1.2cm}|p{4cm}|p{4cm}|}
%		\hline
%		Длина & \multicolumn{2}{c|}{Последовательности} \\ \hline
%		$2$ & $+1 -1$ & $+1 +1$ \\ \hline
%		$3$ & \multicolumn{2}{l|}{$+1 +1 -1$} \\ \hline
%		$4$ & $+1 -1 +1 +1$ & $+1 -1 -1 -1$ \\ \hline
%		$5$ & \multicolumn{2}{l|}{$+1 +1 +1 -1 +1$} \\ \hline
%		$7$ & \multicolumn{2}{l|}{$+1 +1 +1 -1 -1 +1 -1$} \\ \hline
%		$11$ & \multicolumn{2}{l|}{$+1 +1 +1 -1 -1 -1 +1 -1 -1 +1 -1$} \\ \hline
%		$13$ & \multicolumn{2}{l|}{$+1 +1 +1 +1 +1 -1 -1 +1 +1 -1 +1 -1 +1$} \\ \hline
%	\end{tabular}
%\end{table} 
%
%
%
%%%% Далее идет пример оформления ненумерованного списка. Поскольку файл стиля перегружен, то пришлось вводить костыль
%
%М"~последовательности обладают следующими основными свойствами:	
%\renewcommand{\labelitemi}{--}
%\begin{itemize}
%	\item М"~последовательности являются периодическими с периодом $M=2^n-1$, где $n$"--- порядок порождающего полинома $h(x)$;
%	\item количество символов, принимающих значение единица, на длине одного периода М"~последовательности на единицу больше, чем количество символов, принимающих значение ноль;
%	\item сумма по модулю $2$ любой М"~последовательности с её произвольным циклическим сдвигом также является М"~последовательностью;
%	\item периодическая автокорреляционная функция любой М"~последовательности имеет соотношение уровней основного и боковых лепестков, равное $\displaystyle\frac{1}{M}$;
%	\item автокорреляционная функция усечённой М"~последовательности (непериодическая последовательность длиной, соответствующей периоду $M$) имеет соотношение уровней основного и боковых лепестков, близкое к $\displaystyle\frac{1}{\sqrt{M}}$.
%\end{itemize}
\vspace{10mm}
\clearpage


% Цикл формирования лабораторных в сборнике
\newcommand \labnumcount{3} % Номер последней лабы
\newcounter{labnum} % Номер первой лабы
\forloop{labnum}{3}{\not{\value{labnum} > \labnumcount}}{
    \newpage
	\input{Lab-\gettext{\arabic{labnum}}/title.tex}

	\subsection*{Основные теоретические сведения}
	\subimport{Lab-\gettext{\arabic{labnum}}/}{theory.tex}
}

\newpage
\begin{center}%
\section*{Список литературы}
\end{center}%
\addcontentsline{toc}{section}{\bibname}
\printbibliography

\end{document}