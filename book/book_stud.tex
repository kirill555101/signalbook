\documentclass[10pt]{book}
\usepackage{bmstu-lab-book}
\usepackage{forloop}
\usepackage[subpreambles=false]{standalone}
\usepackage[pages=some]{background}
\usetikzlibrary{decorations.pathreplacing}
\usetikzlibrary{arrows,shapes,positioning,shadows,trees}
\usepackage{titlepic}
\usepackage{graphicx}
\usepackage{lastpage}
\usepackage{fancyhdr}
\usepackage{rotating}
\usepackage[font=small]{caption}


%%% Интервал между абзацами, дополняющий обычный интервал между строками
\setlength{\parskip}{0mm}
% Настройка для отображения ответов на контрольные вопросы
\boolfalse{answers} 

% Знаки TradeMark и Registered
\usepackage{textcomp}
\usepackage{xspace}
\let\OldTextregistered\textregistered
\renewcommand{\textregistered}{\OldTextregistered\xspace}%
\let\OldTexttrademark\texttrademark
\renewcommand{\texttrademark}{\OldTexttrademark\xspace}%

% Для регистрации папок с конкретными лабами и определения их порядка в сборнике
\usepackage{etoolbox} 
\newcounter{cnt} 
\newcommand\textlist{} 
\newcommand\settext[2]{% 
\csdef{text#1}{#2}} 
\newcommand\addtext[1]{% 
\stepcounter{cnt}% 
\csdef{text\thecnt}{#1}} 
\newcommand\gettext[1]{% 
\csuse{text#1}} 
\newcounter{colnum} \newcommand\maketabularrow[1]{% 
\setcounter{colnum}{0}% 
\whileboolexpr { test {\ifnumcomp{\value{colnum}}{<}{#1}} }% 
{&\stepcounter{colnum}\thecolnum} } 

% Параметры страниц
\special{papersize=170mm,240mm}
\textheight 170mm % 200-(12+25)*0.35146 = 186.99598
\textwidth 120mm
\headheight13.6pt % = 0.48 mm
\oddsidemargin -10mm
\evensidemargin -10mm
\topmargin -20mm
\usepackage{setspace}
\singlespacing

%%% Для формирования списка литературы BibLaTeX (соритировка по мере появления в тексте)
\usepackage[backend=biber,bibencoding=utf8,sorting=none,maxcitenames=2,bibstyle=numeric,citestyle=numeric]{biblatex}
\addbibresource{lib.bib} % список литературы

\graphicspath{tikz}

% Чтобы ушел заголовок содержания с верхнего колонтитула каждой страницы
% Подробнее: https://www.overleaf.com/learn/latex/Headers_and_footers
\pagestyle{myheadings}

%\usepackage{setspace}\onehalfspacing
%\AtBeginDocument{%
%	\addtolength\abovedisplayskip{1\baselineskip}%
%	\addtolength\belowdisplayskip{-0.1\baselineskip}%
%	\addtolength\abovedisplayshortskip{-0.1\baselineskip}%
%	\addtolength\belowdisplayshortskip{-0.1\baselineskip}%
%}

\usepackage{tocloft}
\setlength{\cftpartindent}{0em}
\setlength{\cftchapindent}{0em}
\setlength{\cftsecindent}{0em}
\setlength{\cftsubsecindent}{0em}
\setlength{\cftsubsubsecindent}{0em}

\fancyhf{} % clear all header and footers
\renewcommand{\headrulewidth}{0pt} % remove the header rule
\fancyfoot[LE,RO]{\thepage} % Left side on Even pages; Right side on Odd pages
\pagestyle{fancy}

\renewcommand{\chapter}[2]{}% убираем стандартный заголовок для библиографии

%%% Объем данных на диске
\newcommand{\filesize}{18,5}

\begin{document}
\renewcommand{\labelitemi}{--}
\renewcommand\contentsname{}% убираем стандартный заголовок для оглавления

% Чnобы не было разрывов после формул
\setlength{\abovedisplayskip}{1.5pt}
\setlength{\belowdisplayskip}{1.5pt}

% Каждая папка с соответствующей лабой должна называться <<Lab-[label]>>, 
% где \addtext{[label]} --- регистрация лабы в сборнике
% Порядок лабораторных работ в сборнике:
\addtext{ma} 	    % 1 - moving average
\addtext{ssf}       % 2 - Synchronous signal filtering
\addtext{fir}    	% 3 - Finite impulse response filtering
\addtext{mf} 	    % 4 - Matched-filtering
\addtext{goertzel} 	% 5 - Goertzel algorithm
\addtext{fq} 	    % 6 - Frequency shifting
\addtext{fmmod} 	% 7 - Phase-Code-Manipulated Signal Receiver
\addtext{hht} 		% 8 - Hilbert–Huang transform

\begin{center}%
\section*{Оглавление}
\end{center}%
\tableofcontents
\thispagestyle{fancy}

% Цикл формирования лабораторных в сборнике
\newcommand \labnumcount{3} % Номер последней лабы
\newcounter{labnum} % Номер первой лабы
\forloop{labnum}{3}{\not{\value{labnum} > \labnumcount}}{
    \newpage
	\input{Lab-\gettext{\arabic{labnum}}/title.tex}

	\subsection*{Основные теоретические сведения}
	\subimport{Lab-\gettext{\arabic{labnum}}/}{theory.tex}
}

\newpage
\begin{center}%
\section*{Список литературы}
\end{center}%
\addcontentsline{toc}{section}{\bibname}
\printbibliography

\end{document}